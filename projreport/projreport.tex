\documentclass[sigconf]{acmart}

%\usepackage{booktabs} % For formal tables
\def\BibTeX{\textsc{Bib}\TeX}
\usepackage{url}
\usepackage{balance}

% Copyright
%\setcopyright{none}
%\setcopyright{acmcopyright}
%\setcopyright{acmlicensed}
\setcopyright{rightsretained}
%\setcopyright{usgov}
%\setcopyright{usgovmixed}
%\setcopyright{cagov}
%\setcopyright{cagovmixed}


% DOI
%\acmDOI{10.475/123_4}

% ISBN
%\acmISBN{123-4567-24-567/08/06}

%Conference
\acmConference[NA]{NA}{2020}{NA}
\acmYear{2020}
\copyrightyear{2020}
%\acmPrice{15.00}


\begin{document}
\title{Gamers Rise Up}
\titlenote{Produces the permission block, and
  copyright information}
%\subtitle{Extended Abstract}
%\subtitlenote{The full version of the author's guide is available as
%  \texttt{acmart.pdf} document}


\author{Anonymous}
\affiliation{
  \institution{Removed}
}

\begin{abstract}
  The abstract should be one or two paragraphs that
  summarize your paper. Abstracts are read independently
  from the rest of the paper so you cannot cite your paper
  or any other papers in it. Study other abstracts in the papers
  you are reading to understand what an abstract should
  really means. {\bf Write the abstract in third person.}

 The abstract should make it clear what your work is about
  understanding research papers and about integrated ideas from
  others. Don't write it as though you did the work done by the
  researchers who did the original research! 

  The abstract is not an introduction or overview, but a summary that
  informs the reader of the context, content and contributions of your
  paper.

\end{abstract}

\keywords{ACM proceedings, \LaTeX, text tagging}

\maketitle


\section{Introduction}
\label{introduction}

Use three or four paragraphs to present an overview of your project.

Provide a roadmap for the remaining sections of the paper. For
example, you can state that Section \ref{design considerations}
presents a discussion of the design and implementation issues you
considered, section \ref{architecture} presents the architecture of
the project, and section \ref{implementation} discusses how you went
about implementing your project.  Section \ref{lessons learned} should
discuss what you learned from this exercise in sufficient depth, and
section \ref{current status and future work} should describe the
current state of the project and what else could be done in the
future.

{\bf Note: This specific file is a generic template so the
  section titles may not fit your specific needs so feel
  free to change them as needed.}

\section{Design Considerations}
\label{design considerations}

Use this section to describe the basic design of your project. Include
the extensions you proposed either in this section or in section
\ref{architecture} or in section \ref{implementation}, wherever it
fits best for your project.

Also review the Related Work in the literature (practice or research)
to place your project in perspective, and what other people have been
doing to address this problem. Make sure your literature survey is
fairly complete, and you must cite your sources correctly per ACM
style guidelines (and of course, you need to use \LaTeX and \BibTeX
correctly).

\section{Architecture}
\label{architecture}

Use this section to describe the overall architecture of your database
engine, and implementation of your
project.

\section{Implementation}
\label{implementation}

Use this section to describe the overall implementation of your
project.

\section{Lessons Learned}
\label{lessons learned}

\subsection{Response to Investigating Teams}
\label{response}

Discuss how you addressed the issues, if any, that were identified by
the investigating teams. Be specific.

\subsection{Other Lessons}
\label{other lessons}

Use this section to describe mistakes you made and corrected (or did
not get a chance to correct including why you didn't). Also describe
what all you learned during the course of this effort; this section,
like the others, plays a critical component in determining your final
grade.

\section{Ethical and Legal Issues}
\label{ethics-legal}

This is not an optional section: it is required. Discuss laws that
apply to your project area, as appropriate. If there are breaches that
impact the area, discuss those in terms of the laws that were broken.

Discuss ethical issues underlying the project area. Provide guidance
to ethical issues relevant to this project area. Use the ACM Code of Conduct
to organize this section.


\section{Current Status \& Future Work}
\label{current status and future work}

Use this section to describe the current status of your work
and what else should be done. Also, discuss what further directions
your work can be taken by others.

\begin{comment}
\subsection{Tables, Figures, and Citations/References}

{\textcolor{red} {\bf This subsection is meant to provide you with
  some help in dealing with figures, tables and citations, as these
  are sometimes hard for folks new to \LaTeX. Your figures, tables and
  citations must be distributed all over your paper (not here), as
  appropriate for your paper.  Please DELETE this subsection
  before you make any submissions!}}



\begin{table}
\centering
\caption{Feelings about Issues}
\label{SAMPLE TABLE}
\begin{tabular}{|l|r|l|} \hline
Flavor&Percentage&Comments\\ \hline
Issue 1 &  10\% & Loved it a lot\\ \hline
Issue 2 &  20\% & Disliked it immensely\\ \hline
Issue 3 &  30\% & Didn't care one bit\\ \hline
Issue 4 &  40\% & Duh?\\ \hline
\end{tabular}
\end{table}


First, note that figures in the term paper must be original, that is,
created by the student: please do not cut-and-paste figures from any
other paper you have read. Second, if you do need to include figures,
they should be handled as demonstrated here. State that
Figure~\ref{SAMPLE FIGURE} is a simple illustration used in the ACM
Style sample document. Never refer to the figure below (or above)
because figures may be placed by \LaTeX{} at any appropriate location
that can change when you recompile your source $.tex$
file. Incidentally, in proper technical writing (for reasons beyond
the scope of this discussion), table captions are above the table and
figure captions are below the figure. So the truly junk information
about flavors is shown in Table~\ref{SAMPLE TABLE}.

\begin{figure}[htb]
\begin{center}
\includegraphics[width=1.5in]{fly.jpg}
\caption{A sample black \& white graphic (JPG).}
\label{SAMPLE FIGURE}
\end{center}
\end{figure}

Finally, citing documents needs to be done properly too. For example,
a paper by Mic Bowman, Saumya K. Debray, and Larry L. Peterson could
be cited as Bowman, Debray, and Peterson~\cite{bowman:reasoning}. A
set of papers could collectively be cited as the literature in this
area consists of several interesting
papers~\cite{braams:babel,clark:pct,herlihy:methodology}. One of the
common types of citations these days is to items only posted on the
Web such as this 2014 CMU SEI webinar by Dormann et al.~\cite{dormann:API}.

You will find the \BibTeX{} entries needed for many papers being cited,
otherwise you can write your own versions easily and add them to the
$report.bib$ file in the folder. There are many sample bibtex
template files that can be used to model your own references.

The list of all references will be generated in the standard ACM Reference
style using \LaTeX{}/\BibTeX{} correctly. Note that you
need to first the following sequence to get the paper
compiled correctly:

\begin{enumerate}
\item {\tt latex} {\em projreport}
\item {\tt bibtex} {\em projreport}
\item {\tt latex} {\em projreport}
\item {\tt latex} {\em projreport}
\end{enumerate}
\end{comment}

\balance
\bibliographystyle{ACM-Reference-Format}
\bibliography{projreport} 

\end{document}
