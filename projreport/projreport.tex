\documentclass[sigconf]{acmart}

%\usepackage{booktabs} % For formal tables
\def\BibTeX{\textsc{Bib}\TeX}
\usepackage{url}
\usepackage{balance}

% Copyright
%\setcopyright{none}
%\setcopyright{acmcopyright}
%\setcopyright{acmlicensed}
\setcopyright{rightsretained}
%\setcopyright{usgov}
%\setcopyright{usgovmixed}
%\setcopyright{cagov}
%\setcopyright{cagovmixed}


% DOI
%\acmDOI{10.475/123_4}

% ISBN
%\acmISBN{123-4567-24-567/08/06}

%Conference
\acmConference[NA]{NA}{2020}{NA}
\acmYear{2020}
\copyrightyear{2020}
%\acmPrice{15.00}


\begin{document}
\title{The Ransomware Report}
\titlenote{Produces the permission block, and
  copyright information}
%\subtitle{Extended Abstract}
%\subtitlenote{The full version of the author's guide is available as
%  \texttt{acmart.pdf} document}


\author{Anonymous}
\affiliation{
  \institution{Removed}
}

\begin{abstract}
  TBD

% Information not rendered, but preserved because it's useful.
\iffalse
  The abstract should be one or two paragraphs that
  summarize your paper. Abstracts are read independently
  from the rest of the paper so you cannot cite your paper
  or any other papers in it. Study other abstracts in the papers
  you are reading to understand what an abstract should
  really means. {\bf Write the abstract in third person.}

 The abstract should make it clear what your work is about
  understanding research papers and about integrated ideas from
  others. Don't write it as though you did the work done by the
  researchers who did the original research! 

  The abstract is not an introduction or overview, but a summary that
  informs the reader of the context, content and contributions of your
  paper.
\fi

\end{abstract}

\keywords{Malware, Ransomware, Cryptography}

\maketitle

\section{Introduction}
\label{introduction}
    As technology continues to become more integral to the day-to-day operations of municipalities and businesses across the world, it becomes increasingly important to analyze the inherent risks that come with conducting business and storing important files in a digital world. One considerable threat to almost all organizations is ransomware: a type of malware that renders users' files inaccessible by encrypting them with a key known only to the attacker. This means the affected user may permanently lose access to important documents and data unless a large sum of money is paid or some other condition set by the malicious party is met. 
    
    In the last quarter of 2019, the average payment requested to release files from ransomware was \$84,116, but that number has been steadily increasing and culminated in a staggering average of \$190,946 in December of 2019. Over 200,000 organizations reported a ransomware infection in 2019, and experts project that this number will only increase in the coming years \cite{popper_2020}. In one high profile incident, the city of New Orleans was forced to declare a state of emergency in response to a December 2019 ransomware attack that crippled many of the city's computer systems and left citizens unable to access basic services like online payment systems and district court appointment scheduling \cite{Winder_2019}. 
    
    New Orleans is just one major example in a startling global trend; ransomware has also plagued large schools, businesses, and even hospitals across the world. The consequences of being infected with this malware are dire, and researchers and cybersecurity experts have yet to propose a general solution to effectively combat ransomware \cite{Norton_2018}. The goal of this paper is to study how ransomware takes control of a test environment in Windows, and determine what countermeasures can be implemented to effectively block this kind of attack or mitigate its damage.
    
    The testing methodology for studying this type of malware is relatively straightforward: First, the subject on the target machine will proceed to access a phishing email or some form of disguised link to a download that will contain the malicious code. Next, once the ransomware code is actually executed, mock files that are designed to simulate user data will be securely encrypted by the malware using the Advanced Encryption Standard (AES) with a key length of 128 bits, and an indication will display informing the user that their access to their files has been blocked. Lastly, the user will need to perform a particular action in order to have the files decrypted, which in the context of this project will involve the subject playing a challenging mini-game to have the files restored. 
    
    In most cases, a ransom in the form of cryptocurrency is demanded by the malicious actor - hence the term \emph{ransomware} \cite{Ransomware_background}. This paper utilizes a local and basic task as an alternative to a cryptocurrency transfer to preserve its focus on the malware and to avoid involving networked transfers. Once deployed and running on a test virtual machine through RIT's RLES service, the malware will be reverse-engineered in an attempt to find a way to detect it prior to execution, and a recovery method through an anti-malware tool will be created in order to recover the test user's files if encryption takes place.
    
    The particular technologies used to develop this malware were chosen to reflect the current state of personal computer (desktops, laptops, etc.) use by businesses, schools, and other large organizations in 2020. The target machine reflects an ordinary and freshly imaged enterprise copy of Windows, which is the predominant non-mobile operating system with over 80\% of the market share \cite{Netmarketshare_2020}. The languages and patterns chosen to develop the ransomware with were modelled off real-world examples of ransomware that have been released under an open source license. The language used to develop the ransomware is Python 3, but because standard Windows installations do not come with a Python interpreter pre-installed, the code needs to be transformed to actually run on the target machine. Before distribution, it is converted to a binary executable file that contains all necessary dependencies and a portable Python interpreter so that the executable is completely independent and able to be run.
    
    The remainder of this paper takes into consideration the practical approaches that will be taken with testing ransomware in a controlled environment. Section \ref{design considerations} details specifically what design elements were crucial to the development of this malware, and what other options were available in order to simulate an attack, but were ultimately either flawed or were outside the scope of this paper. Section \ref{architecture} discusses the structure and format of the project, including the considerations that were chosen to move forward with including operating system and language choice for the ransomware scripts. In detail, Section \ref{implementation} gives a step-by-step account of the process in executing this artificial malware, including how the payload containing the code is created, delivered, and what sort of files are taken over when the code is executed. After considering what the ransomware does, Section \ref{lessons learned} details the reflection process and what worked and did not work according to the original design plan, as well as what techniques helped to achieve the desired outcome for the project. Section \ref{ethics-legal} discusses the ethical and legal consequences of releasing this malware out to the public without explicit and written consent outside of a test environment, and what it could mean for the future of cybersecurity. Lastly, \ref{current status and future work} discusses where the project currently stands in comparison to the project plan, and what work could be done to improve the project in the future.
    
    

\begin{comment}

Use three or four paragraphs to present an overview of your project.

Provide a roadmap for the remaining sections of the paper. For
example, you can state that Section \ref{design considerations}
presents a discussion of the design and implementation issues you
considered, section \ref{architecture} presents the architecture of
the project, and section \ref{implementation} discusses how you went
about implementing your project.  Section \ref{lessons learned} should
discuss what you learned from this exercise in sufficient depth, and
section \ref{current status and future work} should describe the
current state of the project and what else could be done in the
future.

{\bf Note: This specific file is a generic template so the
  section titles may not fit your specific needs so feel
  free to change them as needed.}
\end{comment}

\section{Design Considerations}
\label{design considerations}
TBD
% Information not rendered, but preserved because it's useful.
\iffalse
Use this section to describe the basic design of your project. Include
the extensions you proposed either in this section or in section
\ref{architecture} or in section \ref{implementation}, wherever it
fits best for your project.

Also review the Related Work in the literature (practice or research)
to place your project in perspective, and what other people have been
doing to address this problem. Make sure your literature survey is
fairly complete, and you must cite your sources correctly per ACM
style guidelines (and of course, you need to use \LaTeX and \BibTeX
correctly).
\fi

\section{Architecture}
\label{architecture}
TBD
% Information not rendered, but preserved because it's useful.
\iffalse
Use this section to describe the overall architecture of your database
engine, and implementation of your
project.
\fi

\section{Implementation}
\label{implementation}
TBD
% Information not rendered, but preserved because it's useful.
\iffalse
Use this section to describe the overall implementation of your
project.
\fi

\section{Lessons Learned}
\label{lessons learned}

\subsection{Response to Investigating Teams}
\label{response}
TBD
% Information not rendered, but preserved because it's useful.
\iffalse
Discuss how you addressed the issues, if any, that were identified by
the investigating teams. Be specific.
\fi

\subsection{Other Lessons}
\label{other lessons}
TBD
% Information not rendered, but preserved because it's useful.
\iffalse
Use this section to describe mistakes you made and corrected (or did
not get a chance to correct including why you didn't). Also describe
what all you learned during the course of this effort; this section,
like the others, plays a critical component in determining your final
grade.
\fi

\section{Ethical and Legal Issues}
\label{ethics-legal}

TBD
% Information not rendered, but preserved because it's useful.
\iffalse
This is not an optional section: it is required. Discuss laws that
apply to your project area, as appropriate. If there are breaches that
impact the area, discuss those in terms of the laws that were broken.

Discuss ethical issues underlying the project area. Provide guidance
to ethical issues relevant to this project area. Use the ACM Code of Conduct
to organize this section.
\fi

\section{Current Status \& Future Work}
\label{current status and future work}

TBD
% Information not rendered, but preserved because it's useful.
\iffalse
Use this section to describe the current status of your work
and what else should be done. Also, discuss what further directions
your work can be taken by others.
\fi

\begin{comment}
\subsection{Tables, Figures, and Citations/References}

{\textcolor{red} {\bf This subsection is meant to provide you with
  some help in dealing with figures, tables and citations, as these
  are sometimes hard for folks new to \LaTeX. Your figures, tables and
  citations must be distributed all over your paper (not here), as
  appropriate for your paper.  Please DELETE this subsection
  before you make any submissions!}}



\begin{table}
\centering
\caption{Feelings about Issues}
\label{SAMPLE TABLE}
\begin{tabular}{|l|r|l|} \hline
Flavor&Percentage&Comments\\ \hline
Issue 1 &  10\% & Loved it a lot\\ \hline
Issue 2 &  20\% & Disliked it immensely\\ \hline
Issue 3 &  30\% & Didn't care one bit\\ \hline
Issue 4 &  40\% & Duh?\\ \hline
\end{tabular}
\end{table}


First, note that figures in the term paper must be original, that is,
created by the student: please do not cut-and-paste figures from any
other paper you have read. Second, if you do need to include figures,
they should be handled as demonstrated here. State that
Figure~\ref{SAMPLE FIGURE} is a simple illustration used in the ACM
Style sample document. Never refer to the figure below (or above)
because figures may be placed by \LaTeX{} at any appropriate location
that can change when you recompile your source $.tex$
file. Incidentally, in proper technical writing (for reasons beyond
the scope of this discussion), table captions are above the table and
figure captions are below the figure. So the truly junk information
about flavors is shown in Table~\ref{SAMPLE TABLE}.

\begin{figure}[htb]
\begin{center}
\includegraphics[width=1.5in]{fly.jpg}
\caption{A sample black \& white graphic (JPG).}
\label{SAMPLE FIGURE}
\end{center}
\end{figure}

Finally, citing documents needs to be done properly too. For example,
a paper by Mic Bowman, Saumya K. Debray, and Larry L. Peterson could
be cited as Bowman, Debray, and Peterson~\cite{bowman:reasoning}. A
set of papers could collectively be cited as the literature in this
area consists of several interesting
papers~\cite{braams:babel,clark:pct,herlihy:methodology}. One of the
common types of citations these days is to items only posted on the
Web such as this 2014 CMU SEI webinar by Dormann et al.~\cite{dormann:API}.

You will find the \BibTeX{} entries needed for many papers being cited,
otherwise you can write your own versions easily and add them to the
$report.bib$ file in the folder. There are many sample bibtex
template files that can be used to model your own references.

The list of all references will be generated in the standard ACM Reference
style using \LaTeX{}/\BibTeX{} correctly. Note that you
need to first the following sequence to get the paper
compiled correctly:

\begin{enumerate}
\item {\tt latex} {\em projreport}
\item {\tt bibtex} {\em projreport}
\item {\tt latex} {\em projreport}
\item {\tt latex} {\em projreport}
\end{enumerate}
\end{comment}

\balance
\bibliographystyle{ACM-Reference-Format}
\bibliography{projreport} 

\end{document}
